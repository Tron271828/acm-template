\subsection{C++ 中的字符串类}
\subsubsection{string}
	有些字符串操作本来很简单,但是在 C 语言里就很麻烦。C++ 通过 string 类,在一定程度上缓解了这个问题。string 的头文件为 \header{string}。

	\emph{注意,string 类虽然方便,但是速度很慢!}
	
	\paragraph{输入/输出}
		cin、cout 可以直接输入、输出 string 字符串。如果需要整行读入,可以用 getline 函数:

		\begin{lstlisting}
string st;
cin>>st;

while (getline(cin, st))
{
	cout<<st;
}
		\end{lstlisting}
		
	\paragraph{基本操作}
		string 类相当于 vector<char>,所以 string 可以使用 vector 所支持的大部分操作,例如:

		\begin{lstlisting}
string st="abcde";
cout << st[1];
st.push_back('f');
		\end{lstlisting}
		
		\begin{itemize}
			\item 可以用``+''连接两个字符串。当然,两个字符串中需要至少一个是 string 类型,否则会编译错误。\emph{由于涉及内存分配,不建议使用。}
			\item 用 ==、!= 判断两个字符串是否相等。用比较符号比较两个字符串的大小。
			\item 用 += 运算符或 st.append() 在 st 后面附加字符串。
			\item 用 st.size() 获得字符串 st 的长度。
			\item 用 st.find() 在字符串里寻找子串。如果没找到,函数会返回 string::npos。
			\begin{lstlisting}
size_t p = st.find("hello", 0);  // 第二个参数表示起始位置。
			\end{lstlisting}
			\item 用 st.substr(5, 10) 来截取从第 5 个字符开始、长度为 10 的字符串。如果长度为 string::npos,则一直截取到字符串结束。
			\item 用 st.c\_{}str() 获取一个字符串数组,这样就可以继续用 C 语言的处理方式来处理字符串了。
		\end{itemize}

\subsubsection{stringstream}
	stringstream 可用于从字符串中读取内容。头文件 <sstream>。
	
	\emph{注意,stringstream 的速度非常慢。}
	
	以下是“输入若干行整数,求它们的和”的代码:
	
	\begin{lstlisting}
string st;
while (getline(cin, st))
{
	stringstream ss(st);
	int sum=0, p;
	while (ss>>p) sum+=p;
	cout<<p<<endl;
}
	\end{lstlisting}